% Adatbazisok Laboratorium LaTeX jkv sablon
%
% Keszitette: Szepes Nora, 2017.
% Danyi Bence 2013-mas sablonja alapjan 
%
% XeLaTex kell hozzá a különböző betűtípusok miatt.
%
% A sablon használata:
%  A \fillme reszek kitoltese 
%  1.  A fedlap adatainak kitöltése
%  2.  A feladatok rész kitöltése
% (3.) Megjegyések a méréshez
% 

% Minden egyéb dolgot figyelmen kívül lehet hagyni.
\documentclass[12pt]{article}
\usepackage{fontspec}
\setmainfont[Ligatures=TeX]{Arial}
\usepackage{amssymb,amsmath}
\usepackage{parskip}
\usepackage{graphicx}
\usepackage{moreverb}
\usepackage{array}
\usepackage{titlesec}
\usepackage[left=1.77cm,top=1.27cm,right=1.27cm,nohead,nofoot]{geometry}
\usepackage[magyar]{babel}
\usepackage{xcolor}
%sarga hatter a kitoltendo reszekhez
\usepackage{soul}
\sethlcolor{yellow} 
\newcommand{\fillme}[1]{\emph{\hl{#1}}}


% Értelemszerűen kitöltendő
% A nem szükséges mezők törlendők
\newcommand{\nev}{\fillme{Minta Andrea}}
\newcommand{\neptun}{\fillme{ABCD123}}
\newcommand{\feladat}{\fillme{16 -- BANK}}
\newcommand{\meresvezeto}{\fillme{Másik Ember}}
\newcommand{\meresideje}{\fillme{2011-03-28 8:00}}
\newcommand{\mereshelye}{\fillme{HSZK Q}}
\newcommand{\elerhetoseg}{\fillme{törlendő, ha nem kéri az útmutató}}
\newcommand{\felhasznalonev}{\fillme{törlendő, ha nem kéri az útmutató}}
\newcommand{\jelszo}{\fillme{törlendő, ha nem kéri az útmutató}}
\newcommand{\megoldott}{\fillme{1, 2, 3a, 4}}
% a megoldottnak jelölt feladatok max pontszámának összege pluszpontok nélkül
\newcommand{\pont}{\fillme{2,5}}
% Ugorj a feladatokhoz!
% Itt ne bánts semmit...

\renewcommand{\arraystretch}{1.2}
\renewcommand{\thesection}{\arabic{section}.}
\titleformat
{\section}
{\large\bfseries\itshape}
{\thesection}
{3.8cm}{\vspace{-0.5cm}}
\titlespacing{\section}{-0.5cm}{0pt}{12pt}

\begin{document}
% FEJLEC ELKESZITESE
\begin{center}
\vspace{6pt}
\hspace{-0.5cm}{\Huge\textbf{Mérési jegyzőkönyv -- Adatbázisok Laboratórium}}

\vspace{6pt}
\hspace{-0.5cm}{\Large\emph{\fillme{X}. mérés: \fillme{Mérés neve}}}

\vspace{12pt}
\hspace{-0.5cm}\begin{tabular}[h]{|p{10cm}|p{8cm}|}
\hline {Név:} & \textbf{\nev} \\
\hline {Neptun kód:} & \textbf{\neptun} \\
\hline {Feladat kódja:} & \textbf{\feladat} \\
\hline {Mérésvezető neve:} & \textbf{\meresvezeto} \\
\hline {Mérés időpontja:} & \textbf{\meresideje} \\
\hline {Mérés helyszíne:} & \textbf{\mereshelye} \\
\hline {A működő alkalmazás elérhetősége:} & \textbf{\elerhetoseg} \\
\hline {Felhasználónév:} & \textbf{\felhasznalonev} \\
\hline {Jelszó:} & \textbf{\jelszo} \\
\hline {Megoldott feladatok:} & \textbf{\megoldott} \\
\hline {Elérhető pontszám (a megoldottnak jelölt feladatok max pontszámának összege pluszpontok nélkül):} & \textbf{\pont p} \\
\hline
\end{tabular}
\end{center}

% FELHASZNALOI UTMUTATO JDBC-HEZ
\vspace{24pt}
\hspace{-0.5cm}{\Large \textbf{Felhasználói útmutató}}

\vspace{-0.5cm}
\hspace{-0.5cm}\rule{\textwidth}{0.4pt}

\fillme{Csak a JDBC mérés esetén szükséges, egyébként törlendő. Kb. fél oldal terjedelmű használati segédlet, amely a laikus felhasználók számára is érthető nyelven leírja az alkalmazás működését.}
\newpage

% FELADATOK MEGOLDASANAK LEIRASA
\vspace{24pt}
\hspace{-0.5cm}{\Large \textbf{Mérési feladatok megoldása}}

\vspace{-0.5cm}
\hspace{-0.5cm}\rule{1.02\textwidth}{0.4pt}


% ITT KEZDODNEK A FELADATOK
% ERDEMES MINDNE FELADATOT UJ OLDALRA RAKNI
\section{Feladat}

\textbf{\fillme{Magyarázat}}

\fillme{Ide kerül a feladat megoldásának magyarázata, esetleges képekkel, forráskód részletekkel. Lehet menet közben kód az egyes feladatok megoldásában, erre használjuk a kód stílust! Forráskódot képként nem szúrhatsz be!}
% Ha szóközökkel indentálsz, akkor jó a sima verbatim környezet is
\begin{verbatimtab}
	SELECT *
	FROM CODE
	WHERE
	EGYIK = 5
	AND MASIK = 6;
	STB. STB.
\end{verbatimtab}


\textbf{\fillme{Tesztelés menete}}

\fillme{A feladatot így és így teszteltem. Parancsokat szövegként kell beilleszteni, egyéb esetben screenshot és/vagy konzolból kimásolt kimenet.}

\textbf{\fillme{Példaadatok}}

\fillme{Ha a feladat teszteléséhez szükség van példaadatokra, azt itt tüntessük fel.}



\newpage
\section{Feladat}


\textbf{\fillme{Magyarázat}}

\fillme{Ide kerül a feladat megoldásának magyarázata, esetleges képekkel, forráskód részletekkel. Lehet menet közben kód az egyes feladatok megoldásában, erre használjuk a kód stílust! Forráskódot képként nem szúrhatsz be!}
% Ha szóközökkel indentálsz, akkor jó a sima verbatim környezet is
\begin{verbatimtab}
	SELECT *
	FROM CODE
	WHERE
	EGYIK = 5
	AND MASIK = 6;
	STB. STB.
\end{verbatimtab}


\textbf{\fillme{Tesztelés menete}}

\fillme{A feladatot így és így teszteltem. Parancsokat szövegként kell beilleszteni, egyéb esetben screenshot és/vagy konzolból kimásolt kimenet.}

\textbf{\fillme{Példaadatok}}

\fillme{Ha a feladat teszteléséhez szükség van példaadatokra, azt itt tüntessük fel.}


% VELEMENY A MERESROL
\newpage
\vspace{24pt}
\hspace{-0.5cm}{\Large \textbf{Vélemény(ek) a mérésről}}

\vspace{-0.5cm}
\hspace{-0.5cm}\rule{\textwidth}{0.4pt}
% Ide jöhetnek a megjegyzések

\fillme{Vélemény, építő jellegű kritika. Ha nincs, törlendő.}
\end{document}
